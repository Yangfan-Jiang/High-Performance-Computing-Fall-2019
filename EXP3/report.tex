\documentclass{article}
\usepackage[UTF8]{ctex}
\usepackage{fancyhdr}
\usepackage{extramarks}
\usepackage{amsmath}
\usepackage{amsthm}
\usepackage{amsfonts}
\usepackage{tikz}
\usepackage[plain]{algorithm}
\usepackage{algpseudocode}

\usepackage{fontspec}

\usepackage{graphicx}
\usepackage{subfigure}

\usepackage{geometry}
\geometry{a4paper,scale=0.70}


\usepackage{listings}
\usepackage{xcolor}

\usetikzlibrary{positioning}


\lstset{
    numbers=left, 
    numberstyle= \tiny, 
    basicstyle = \fontspec{Consolas},
    keywordstyle = \fontspec{Consolas Bold},
    commentstyle = \fontspec{Consolas Italic},
    keywordstyle= \color{ blue!70},
    commentstyle= \color{red!50!green!50!blue!50}, 
    frame=shadowbox, % 阴影效果
    rulesepcolor= \color{ red!20!green!20!blue!20} ,
    escapeinside=``, % 英文分号中可写入中文
    xleftmargin=0.5em,xrightmargin=0.5em, aboveskip=0.5em,
    framexleftmargin=2em,
    language=C,
    showstringspaces=false
} 


\linespread{1.1}

\pagestyle{fancy}
\lhead{\hmwkAuthorName}
\chead{\hmwkTitle }
%\rhead{\firstxmark}
\lfoot{\lastxmark}
\cfoot{\thepage}

\renewcommand\headrulewidth{0.4pt}
\renewcommand\footrulewidth{0.4pt}

\setlength\parindent{0pt}


\newcommand{\hmwkTitle}{高性能编程实验\#3}
\newcommand{\hmwkAuthorName}{\text{姜洋帆 17341068}}


\title{
    \centering\textbf{ \hmwkTitle}
    \author{\hmwkAuthorName}
}

\renewcommand{\part}[1]{
    \textbf{\large Part \Alph{partCounter}}
    \stepcounter{partCounter}\\
}





\begin{document}

\maketitle

\section{Tasks}
多线程协作实验
\begin{itemize}
    \item pthread线程间的同步函数的实现
    \item 多线程矩阵分块乘法
    \item 列出给出不同规模的加速比和效率
\end{itemize}

\section{实现pthread线程同步函数}
按照实验要求,分别使用两种方式来实现同步函数,关键代码如下(这里只列出了代码的关键部分,完整代码见报告附件)
\newline
\textbf{忙等待}
\begin{lstlisting}
int counter = 0;
pthread_mutex_t barrier_mutex;

void* p_matrix_multi(void *k) {

    /* parallel matrix multiply */

    pthread_mutex_lock(&barrier_mutex);
    counter++;
    pthread_mutex_unlock(&barrier_mutex);
    while(counter < THREAD_NUM);

    ...
    // Synchronization before this operation
    /* reduction results of each thread */
    ...
}
\end{lstlisting}

\textbf{阻塞等待}

\begin{lstlisting}
sem_t count_sem;
sem_t barrier_sem;

pthread_mutex_t barrier_mutex;

void* p_matrix_multi(void *k) {

    /* parallel matrix multiply */

    sem_wait(&count_sem);
    if (counter == THREAD_NUM-1) {
        counter = 0;
        for(int k=0; k < THREAD_NUM-1; k++)
            sem_post(&barrier_sem);
        sem_post(&count_sem);
    } else {
        counter++;
        sem_post(&count_sem);
        sem_wait(&barrier_sem);
    }

    ...
    // Synchronization before this operation
    /* reduction results of each thread */
    ...
}
\end{lstlisting}
\newline
\textbf{注:这各阻塞函数是根据老师实验课提供的PPT编写的,但是实际上存在bug
,counter重置似乎有点问题,会导致该同步只能使用一次}
\newline
\textbf{测试函数}
为了验证并行矩阵分块算法正确性,使用verify函数进行验证(比较并行算法与串行算法结果的差异)
\begin{lstlisting}
bool verify() {
    for(int i=0;i < N; i++)
        for(int j=0; j < M; j++) {
            if(fabs(C1[i][j]-C2[i][j]) > 0.0001)
                return false;
        }
    return true;
} 
\end{lstlisting}

\textbf{经过测试,两种同步方式实现均能保证并行算法得到正确结果}

\section{实现分块矩阵乘法}
\subsection{Method}
根据实验要求实现分块矩阵乘法,每个线程负责实现一个分块矩阵的乘法,全部线程完成任务后,需要进行一次同步(使用上面实现的同步方法),然后进行归并求和。
\subsection{Codes}
\textbf{使用pthread实现矩阵分块乘法,主要的运算操作代码入下:}
\begin{lstlisting}
void* p_matrix_multi(void *k) {
    // [block_m, block_n, block_s]          20 * 2 * 5
    int num_proc = *(int *)k;

    int tmp_m = num_proc / (block_n*block_s);
    int tmp_n = (num_proc / block_s) % block_n;
    int tmp_s = num_proc % block_s;

    for(int i=tmp_m*block_size_m; \
        i < tmp_m*block_size_m+block_size_m; i++)
    for(int j=tmp_n*block_size_n; \
        j < tmp_n*block_size_n+block_size_n; j++)
    for(int s=tmp_s*block_size_s; \ 
        s < tmp_s*block_size_s+block_size_s; s++) {
        private_C[num_proc][i-tmp_m*block_size_m][j-tmp_n*block_size_n]\
         += A[i][s]*B[s][j];
    }

    // Synchronization before this operation
    // reduction
}
\end{lstlisting}
其中private\_C本意为线程的私有变量,代表每个线程的运算结果,这里为了方便操作,直接使用全局数组,每个线程操作不同的第一维,所以本质上还是相互间没有依赖
,不同线程间的任务不会产生race condition.

\textbf{
reduction方法:采用两种方式进行结果的reduction,分别是由一个线程求和以及多线程树形求和,关键代码如下
}
\newline
\textbf{单个线程求和}
\begin{lstlisting}
if(num_proc%block_s == 0) {
    for(int k=num_proc; k < num_proc+block_s; k+=1)
    for(int i=tmp_m*block_size_m; i < tmp_m*block_size_m+block_size_m; i++)
    for(int j=tmp_n*block_size_n; j < tmp_n*block_size_n+block_size_n; j++)
        C2[i][j] += private_C[k][i-tmp_m*block_size_m][j-tmp_n*block_size_n];
}
\end{lstlisting}

\textbf{树形求和}
\newline
这里采用的策略是,每次与有剩下数据的相邻线程进行求和,第一次从偶数线程开始,获取相邻的奇数线程数据进行归并。每一次(每一层)求和结束后,需要通过
同步操作防止race condition,这里选择使用阻塞等待的同步方式。
由于老师PPT提供的非阻塞同步有bug,导致只能使用一次同步函数,所以树形求和每层求和结束后的同步使用pthread库函数pthread\_barrier\_wait来实现。
\newline
数据所在线程的示意情况如下:
\newline
[0,1,2,3,4,5,6,7,8] ==> [0,2,4,6] ==> [0,4] ==>[0]
\begin{lstlisting}
int step = 2;
/* tree reduction */
while(step < 2*block_s) {
    if((num_proc%block_s)%step==0 && \
        (num_proc+(step/2))%block_s < block_s && \
        num_proc/block_s == (num_proc+step/2)/block_s) {
        pthread_mutex_lock(&barrier_mutex);
        pthread_mutex_unlock(&barrier_mutex);
        for(int i=tmp_m*block_size_m; \
        i < tmp_m*block_size_m+block_size_m; i++) {
            for(int j=tmp_n*block_size_n; \
            j < tmp_n*block_size_n+block_size_n; j++) {
                if(step >= block_s) {                       
                    C2[i][j] =\
                    private_C[num_proc][i-tmp_m*block_size_m]\
                        [j-tmp_n*block_size_n] + \
                    private_C[num_proc+(step/2)]\
                    [i-tmp_m*block_size_m][j-tmp_n*block_size_n];
                } else {
                    private_C[num_proc]\
                    [i-tmp_m*block_size_m][j-tmp_n*block_size_n] += \
                    private_C[num_proc+(step/2)]\
                    [i-tmp_m*block_size_m][j-tmp_n*block_size_n];
                }
            }
        }
    }
    
    pthread_barrier_wait(&barrier);
    if(num_proc == 0)
        step *= 2;
    pthread_barrier_wait(&barrier);
}
\end{lstlisting}

两种求和以及两种同步经过测试,结果正确
\begin{figure}[htpb]
    \centering
    \includegraphics[scale=0.65]{1.png}    
    \caption{TEST}
\end{figure}



\subsection{加速比和效率}
在集群上进行实验,观察不同规模输入下,不同线程数和加速比之间的关系,为了方便比较,统一使用非阻塞同步、树形线程归并的代码进行实验
\begin{table}[htbp]  
    \centering
    \caption{加速比/效率}  
    \begin{tabular}{cccccccc}
    \hline  
    Size&1&2&4&8&16&32&64 \\  
    \hline
    128&0.59/0.59&1.09/0.54&2.16/0.54&3.24/0.40&4.82/0.30&3.75/0.11&3.35/0.05 \\  
    256&0.54/0.54&1.06/0.53&1.96/0.49&3.20/0.40&6.06/0.37&6.44/0.20&12.58/0.19 \\  
    512&0.55/0.55&0.91/0.45&1.93/0.48&5.69/0.71&9.20/0.57&10.31/0.32&14.74/0.23 \\
    1024&0.56/0.56&1.06/0.53&1.93/0.48&4.07/0.50&7.42/0.46&10.89/0.34&12.02/0.18 \\
    2048&0.52/0.52&0.91/0.45&1.61/0.40&9.54/1.19&12.83/0.80&30.65/0.95&35.61/0.55 \\
    4096&0.50/0.50&1.13/0.56&1.64/0.41&4.29/0.53&8.47/0.52&10.67/0.26&51.53/0.80 \\
    \hline  
    \end{tabular}  
\end{table}  
\newline
可以看出,大致上的情况为,问题输入规模越大,加速比以及效率越高。对同一个问题而言,线程数越多,效率越低。
但在实验中出现个别异常数据,比如4096规模下32核运行的结果,加速比和效率不如更小规模(2048)相同线程
的情况;2048规模下8线程结果,效率大于1.
\newline 目前重复了几次实验,大致都会得到这几项特殊的结果(结果都通过verify函数的验证),不知道什么原因,猜测可能和
矩阵划分以及CPU cache参数有关。。。可能因为某种原因(矩阵划分方式等等)导致cache命中率出现问题
\newline

\section{优化思考}
\begin{itemize}
    \item 通过适当的划分矩阵,使得数据贴近cache,可以减少cache miss,降低访存压力,达到加速的效果
    \item 当矩阵划分过小或者在某些数据划分边界会出现伪共享
    \item 可以在最初构造AB矩阵时通过数据填充等方式,避免伪共享,也可以给每个线程分配私有的空间,再进行操作,但是会增加赋值开销以及空间开销
\end{itemize}

\section{Note}
\textbf{完整代码和脚本文件见附件,在集群上测试时,
直接执行./compile.sh即可,会编译并执行pbs脚本提交作业,输出
结果包含并行程序用时、串行程序用时、线程数、问题规模以及验证函数
的结果}


\end{document}
